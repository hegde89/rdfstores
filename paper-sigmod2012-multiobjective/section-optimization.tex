\section{Dynamic Programming Based Optimization}
\label{sec:opt}
In this section we propose how to adopt the dynamic
programming (DP) solution~\cite{selinger_access_1979} to the 
multi-objective Linked Data query optimization problem. 

DP for query optimization works in a
bottom-up fashion, constructing the query plan starting from the
leaves, which are scan operators to access
relations. DP is is used to deal with the
exponentially large search space of possible query plans. It takes
advantage of the \emph{optimal substructure} of the query optimization
problem, i.e., the optimal query plan can be constructed from
optimal subplans. Non-optimal subplans can be
discarded during the process to reduce the search space.


Applied to Linked Data query processing, we propose to construct access plans $P(t)$ for every triple patterns $t \in Q$. These \emph{atomic plans} are then successively combined 
%using join operators 
to create \emph{composite plans} for larger subexpressions $T \subseteq Q$. 
For instance, to construct a query plan for the expression $T=t_1\Join t_2$, the optimizer may consider all possible pairs $\{(p1,p_2), p_1 \in P(t_1),p_2 \in P(t_2)\}$ as possible combination of plans. When combining two plans $p_1,p_2$ to form a new plan $p$, we
write $p = \mathtt{cmb}(p_1,p_2)$. At each stage, the optimizer 
try to reduce candidates subplans by discarding those that cannot be part of an optimal solution. That is, before constructing plans for larger subexpressions the optimizer creates $P^+(T) \subseteq
P(T)$, the set of optimal plans for every subexpression $T$. 
%In the case of Linked Data
%query processing, the optimality of a plan is determined according to
%multiple objectives. 

%, i.e., the set of optimal plans is the set of
%Pareto-optimal plans, $P^+(T) = P^*(T)$.

In the following, we firstly discuss how to estimate the optimality of atomic plans as well as composite plans for any expressions $T \subseteq Q$. Then we discuss the main problems that arise when applying the DP solution to this problem of multi-objective Linked Data query optimization.  Because query plans are no longer required to produce all results, we will
discuss a relaxation of the comparability constraint. Then, we study the effect of operator sharing on query optimization and introduce upper and lower bounds on plan
costs. Finally, we prove that the multi-objective query optimization problem still has optimal
substructure and that the dynamic programming solution constructs the
optimal solution, i.e. the skyline of query plans.

\subsection{Estimating Cost and Cardinality of Plans}
\label{sec:estimation}
For the presented structure of a Linked Data query plan and its operators, many existing techniques can be used to systematically estimate cost \cite{stocker_sparql_2008,neumann_scalable_2009,huang_selectivity_2010}. An essential factor for cost estimation is cardinality. While not only input cardinality but also output cardinality may be used for estimating the cost of some operators, optimizing based on cost alone does not guarantee that the resulting plans are also optimal with respect to output cardinality. 
%Instead of cardinality, the work discussed here is also applicable to other 
%In fact, not 
%%Further, other optimization objectives such as 
%only cost and cardinality but other objectives such as quality and relevance, which may have even smaller overlap with cost, have 
%Thus, estimating the optimality of a plan requires 
%For certain operators, input cardinality is not the same as 

%In this case, not only the cost but also the output cardinality are associated with query operators. 
We will now discuss straightforward estimates that will be needed in this work (and refer the readers to more specific work on cost and join size estimations for more advanced techniques \cite{stocker_sparql_2008,neumann_scalable_2009,huang_selectivity_2010}):

\textbf{Operators.} The output cardinality of the source scan operator is the same as the size of the source, i.e. $card(scan_d) = |T^d|$. 
% If the source is not locally available, it will be
%retrieved remotely. 
%A Linked Data source may contain arbitrary
%data and may therefore contain inputs for other triple
%patterns. If the source scan is an input for more than one operator,
%the data will only be retrieved once and will then either be kept in a
%local buffer for subsequent scans or, in the case of push-based
%execution, immediately be pushed to all subsequent operators.
This source size statistics can be directly obtained from the source index discussed before. 
For union, cardinality is the sum of the cardinalities of its inputs: $card(\cup(I_1,...,I_n)) = \sum_{i=1}^n card(I_i)$. 
The cardinality for selection and join depends on selectivity estimates $sel(\cdot)$, i.e. $card(\sigma_t(T_d)) = sel(t) \times |T_d|$ and $card(t_i \Join t_j) =
sel(t_i \Join t_j) \times card(t_i) \times card(t_j)$, respectively. 
%\textbf{Join.} The output cardinality of a join between triples obtained for the two patterns $t_i,t_j$ is given by its selectivity: 
Costs for scan, selection, union and join are $cost(scan_d) = h_s \times |T^d|$, $cost(\sigma_t(T_d))=h_\sigma \times |T_d|$, $cost(\cup) =
h_\cup \times card(\cup)$, $cost(\Join) =
h_\Join \times card(\Join)$. Thus, cost is assumed to be proportional to cardinality but different weights $h_s,h_\sigma,h_\cup,h_\Join$ are used for different operators. The weight factor $h_\Join$ for instance, depends on the
join algorithm employed. As in previous work on Linked Data query processing, we use symmetric hash join~\cite{ladwig_linked_2010,sihjoin_2011}. In case of operator sharing, separate cost
models for the first source scan (when the data is retrieved over the
network) and subsequent scans (when the data has already been
retrieved) are used. We use $cost_2(scan_d) = (1 - b) \times cost_1(scan_d)$, where $cost_1$ denotes first time cost, $cost_2$ stands for cost for each subsequent scan, and $b$
is a parameter to control the benefit achievable through operator sharing.


\textbf{Atomic Plan.} The cardinality of an access plan $p(t)$ is captured by its root node, i.e. $card(p(t)) = card(\cup_t)$. Its cost is calculated as the sum of the cost of its nodes. Source scan operators are marked after first time usage so that the right cost model can be determined for this calculation. 

\textbf{Composite Plan.} Composite plans capture the joins between results obtained for several triple patterns (outputs of access plans). Thus, for an expression $T = t_i\Join t_j$, $card(p(T))  = card(t_i\Join t_j)$ and $cost(p(T))  = cost(t_i\Join t_j)$.


\subsection{Comparability}
\label{sec:comparability}
% The dynamic programming algorithm works in a bottom-up fashion,
% constructing the query plan starting from the leaves, which in the
% classic problem are scan operators to access relations. Dynamic
% programming is used to deal with the exponentially large search space
% of possible query plans. It takes advantage of the optimal
% substructure of the classic query optimization problem, i.e., the
% optimal query plan can be constructed from optimal sub-plans. This
% enables the optimizer to prune non-optimal plans at each stage and
% only retain optimal sub-plans, reducing the search space
% significantly.

Pruning suboptimal plans is an essential part of the dynamic programming solution to
query optimization. 
%However, the optimizer must take care not to
%discard plans that may be part of an overall optimal solution, which
%then would no longer be found. 
For this, the notion of \emph{comparability} was introduced, which is an equivalence relation $\sim$ over plans. It determines which plans are comparable, based on which  
% according
%to some pre-defined properties. 
the optimizer can decide which plans are suboptimal. The optimizer can prune all but the optimal plans for each of the equivalence classes that are induced by $\sim$. 

In the traditional setting, atomic operators and plans comprising them are \emph{comparable when they produce the same output}. This comparability relation is applicable because input relations are determined by the query such that operators used to process them produce the same output and vary only with regard to cost. The optimizer than chooses how to process data (e.g. table or index
scan) based on cost estimates. In Linked Data query processing, however, the selection of
sources (represented by source scan operators) is part of query
optimization. Thus, the optimizer decides both \emph{what and how data shall be processed}. 
If we apply the comparability concept as defined
previously, each unique combination of source scan operators may yield different results and thus, would be represented by a separate equivalence class of query plans. Given there
are potentially hundreds of Linked Data sources for a single query, this may result in a search space where query optimization is no longer affordable.

However, we note that given the objectives here are cardinality and cost, we are not interested in which results but how many results will be produced. 
%is no longer required, 
Accordingly, a relaxation of this comparability relation can be employed that enables the optimizer to prune plans
more aggressively.

\begin{definition}
  \label{def:comparability}
  Two query plans $p_i, p_j$ are \emph{comparable} if they produce results for the same expression, i.e. $p_i(T_i) \sim  p_j(T_j)$ if $T_i = T_j$. 
\end{definition}

This relaxation means that comparable plans produce the same type of results (bindings for the same pattern), but may vary in the number as well as the actual results produced (different bindings). 

Besides results, other aspects have been incorporated into the comparability relation. For instance, plans may be considered comparable when they produce same results (same type of results), and these results are ordered \dtr{cite}. 
%when considering orders: a non-optimal sub-plan might produce ordered output, which can
%be efficiently used by sort-merge joins later in the query plan and
%should therefore not be discarded at the first possibility.
%Considering other properties such as interesting orders, is of course
%possible, but 
Besides this aspect of interesting orders, the inclusion of other objectives such as relevance and quality would also require a different notion of comparability.  
%For clarity, we omit additional constraints that may be added to. 

\subsection{Monotonicity}
\label{sec:sharing}
% As discussed, the query optimization algorithm based on dynamic
% programming constructs optimal plans for a given query $Q$ by starting
% with plans for single triple patterns $t \in Q$ and successively
% combining plans $P(T_1), P(T_2)$ for subexpressions $T_1,T_2 \subset
% Q, T_1 \cap T_2 = \emptyset$ to create plans $P(T)$ for subexpressions
% $T = T_1 \cup T_2$. In order to minimize the search space when
% creating plans $P(T)$, we prune plans from $P(T_1), P(T_2)$ that
% cannot be part of an optimal plan for $T$.

% In query optimization for relational databases, a plan is usually
% considered to be optimal if it has the lowest cost. Therefore it is
% possible to prune all but one plan for each equivalence class (as
% defined in the previous section) to obtain the set of optimal plans
% $P^*(T)$ for a given expression $T$. For optimization of Linked Data
% queries, however, there are two reasons why this is not as simple: 1)
% we no longer use a single dimension (such as cost) to assess query
% plans, but multiple dimensions (cost and cardinality) and 2) we employ
% operator sharing, which means that the cost function is no longer
% monotonic with regard to combining query plans. We tackle the problem
% of constructing Pareto-optimal query plans in the next section. Here,
% we consider each dimension separately and examine the effect of
% operator sharing on query optimization using dynamic programming.

A central requirement for the DP solution here is that the scoring function must be \emph{monotonic} with respect to plan combination. 
%Without this, the problem no longer has optimal substructure \todo{find something we
%  can cite for this}. 
Only then, subplans can be pruned because it can be guaranteed that they cannot be part of optimal plans. Here, plans have to be compared with respect to different objectives. We will now discuss the monotonicity of the functions employed for cost and cardinality. 
%Given monotonic functions for every 
%This still holds in the context of

\textbf{Cardinality.} Atomic plans are combined to capture joins between results. The monotonicity of the cardinality function can be established because the cardinality function for join is monotonic:

\begin{lemma}
  Given a query $Q$, let $T,T' \subset Q$ be two subexpressions of
  $Q$, such that $T \cap T' = \emptyset$. Let $p_1,p_2 \in P(T)$ and
  $p' \in P(T')$ be plans for $T$ and $T'$. Then we have $card(p_1) \leq
  card(p_2) \Rightarrow card(\mathtt{cmb}(p_1,p')) \leq
  card(\mathtt{cmb}(p_2,p'))$.
\end{lemma}
\begin{proof}
  % The output cardinality of combined plan is determined by the input
  % cardinality (i.e., the output cardinality of the two combined plans)
  % and the selectivity of the join operator used to combined the two
  % plans.
  The plan combinations above capture the expression $T \Join T'$. According to the function $card(T \Join T')$, we can write the condition in
  the theorem as $card(p_1) \leq card(p_2) \Rightarrow card(p_1)
  \times card(p') \times sel(T \Join T') \leq card(p_2) \times card(p') \times
  sel(T \Join T')$. This is true due to monotonicity of multiplication.
\end{proof}

\textbf{Cost.} For cost estimation, operator sharing is taken into
account. Because the costs of first and subsequent scans vary, the
cost of the source scan operator changes when a plan is combined with
another plan that shares that operator.
%Monotonicity is not clear which may cause
%the optimizer to miss the overall optimal plan.
Suppose we have two plans $p,p'$ for the subexpression $T \subset Q$
and a plan $p_t$ for a triple pattern $t$ such that $Q = T \cup
\{t\}$, and $cost(p) > cost(p')$. The optimizer would consider $p'$ to
be the optimal plan for $T$ and discard $p$ to form
$P^+(T)=\{p'\}$. Now, because of operator sharing it is possible that
the cost of the combination of two plans is less than the sum of the
cost of the two combined plans, i.e. it is possible that
$cost(\mathtt{cmb}(p,p_t)) < cost(\mathtt{cmb}(p',p_t))$ if $p$ and
$p_t$ share the same source such that the cost of $p_t$ when combined
with $p$ is much lower than the cost of $p_t$ that is combined with
$p'$. In this case, $p'$ should not be part of $P^+(T)$.

% To avoid this problem, We need to make sure that the cost and
% cardinality functions are \emph{monotonic} with regard to plan
% combination, meaning that the following conditions must hold:
% \[ cost(p_1) \leq cost(p_2) \Rightarrow cost(\mathtt{cmb}(p_1,p'))
% \leq cost(\mathtt{cmb}(p_2,p')) \]
% \[card(p_1) \leq card(p_2) \Rightarrow card(\mathtt{cmb}(p_1,p'))
% \leq card(\mathtt{cmb}(p_2,p'))\]
% \todo{should we add a proof that the monotonicity of the card function
%   is not violated when employing operator sharing?}

\textbf{Cost Bounds for Partial Plans.} In order to this effect of
operator sharing into account when calculating the cost of a partial
plan $p$, we define upper and lower bounds for $p$ based on larger
plans that use $p$ as subplans:

\begin{definition}[Lower and Upper Bound Cost]
  \label{def:bounds}
  Given a query $Q$, the subexpressions $T \subset Q$, $T' = Q
  \setminus T$, a plan $p \in P(T)$, and let $P^p(Q) \subseteq P(Q)$
  be the set of all plans for $Q$ that are constructed as combinations
  of $p$ and plans in $P(T')$: $P^p(Q) = \{\mathtt{cmb}(p,p') | p' \in
  P(T')\}$.  Then, we have \emph{lower bound cost} for $p$ as
  $cost_L^Q(p)= MAX\{cost(p)| p \in P^p(Q)\}$ and \emph{upper bound
    cost} for $p$ as $cost_L^Q(p)= MIN\{cost(p)| p \in P^p(Q)\}$.
\end{definition}

Intuitively, a plan $p_i$ for a subexpression $T$ of $Q$ is ``worse''
in terms of cost than another plan $p_j$ for $T$, if all plans for $Q$
that are based on $p_i$ have higher cost than all plans for $Q$ that
are based on $p_j$, i.e., if $cost_L^Q(p_i) > cost_U^Q(p_j)$. Based on
these bounds, we can establish the monotonicity of plan cost with
respect to plan combination as follows:
  
\begin{lemma}
  Let $T,T' \subset Q$ be two subexpressions of $Q$ such that $T \cap
  T' = \emptyset$ and $p_1,p_2 \in P(T)$ and $p' \in P(T')$ be plans
  for $T$ and $T'$, respectively. We have
    \[ cost^Q_U(p_1) \leq cost^Q_L(p_2) \Rightarrow
    cost^Q_U(\mathtt{cmb}(p_1,p')) \leq
    cost^Q_L(\mathtt{cmb}(p_2,p')) \]
\end{lemma}
\begin{proof}
  This follows directly from the definition of $cost^Q_L$ and
  $cost^Q_U$ in Def.~\ref{def:bounds}. Any plan for $Q$ that is
  constructed as a combination of $p'_1 = \mathtt{cmb}(p_1,p')$, i.e.,
  any plan in $P^{p'_1}(Q)$, is also a combination of $p_1$ (because
  $p'_1$ is a combination of $p_1$). From this follows that
  $P^{p'_1}(Q) \subseteq P^{p_1}(Q)$ and $cost^Q_U(p'_1) \leq
  cost^Q_U(p_1)$. For $p_2$ and $p'_2 = \mathtt{cmb}(p_2,p')$ we get
  in the same way that $cost^Q_L(p'_2) \geq cost^Q_L(p_2)$. With this
  we see that $cost^Q_U(p_1) \leq cost^Q_L(p_2) \Rightarrow
  cost^Q_U(p'_1) \leq cost^Q_L(p'_2)$, meaning the lemma is true.
\end{proof}

Based on these results for cardinality and cost monotonicity, we now
refine the dominance relation to make it applicable to subplans,
i.e. plans for strict subexpressions of $Q$:

\begin{theorem}
  \label{def:dominates_bound}
  Given a query $Q$, a subexpression $T \subset Q$ and two plans
  $p_1,p_2$ for $T$, $p_1 > p_2$ if
  $card(p_1) \geq card(p_1) \wedge cost_U^Q(p_1) \leq cost_T^Q(p_2)
  \wedge (card(p_1) > card(p_2) \vee cost_U^Q(p_1) < cost_T^Q(p_2))$.
\end{theorem}

This is the main result needed for pruning. A subplan is suboptimal and thus can be pruned if it is dominated in the sense specified above. 

% With the lower and upper bounds we can reestablish the monotonicity of
% plan costs, however in a less restrictive form:

% \begin{theorem}
%   Given two plans $p_1,p_2$ for a subexpression $T$ and a plan $p'$
%   for a subexpression $T'$, such that $T \cap T' = \emptyset$, the
%   following condition holds:
%   \[ cost_U^{T\cup T'}(p_1) \leq cost_L^{T\cup T'}(p_2) \Rightarrow 
%   cost(\mathtt{cmb}(p_1,p') \leq cost(\mathtt{cmb}(p_2,p')) \]
% \end{theorem}
% \begin{proof}
%   \todo{proof}
% \end{proof}

\textbf{Cost Bound Estimation.} In the form of
Definition~\ref{def:bounds}, the calculation of the lower and upper
bounds of a plan $p$ requires constructing all plans based on
$p$. This is of course very cost intensive and defeats the purpose of
pruning. We therefore try to estimate the bounds by determining the
maximal possible benefit that is achievable through operator sharing
for a particular plan. 

As the source scan operator is the only shareable operator, we can
calculate the maximal possible benefit for a particular plan $p$ by
using the source index to retrieve all sources for triple patterns not
covered by $p$ and aggregating the benefit for all sources that are
also used by $p$.

We only use the maximal benefit when comparing two plans $p,o$ to decide
whether one plan dominates the other. In this case, we can obtain more
a more precise benefit $p$ by only aggregating the benefit for sources
that are not part of $o$. We therefore define the maximal benefit of a
plan $p \in P(T)$ in terms of another plan $o \in P(T)$:

\begin{definition}
  Given a query $Q$, a source index $I$, and two query plans $p,o \in
  P(T), T \subset Q$, let $S_p,S_o$ be the sets of sources retrieved
  by source scan operators in $p,o$, respectively. The maximal benefit
  of $p$ w.r.t $o$ defined as $m_o(p) = \sum_{t \in Q \setminus T}
  \sum_{(d,r_d) \in I(t), d \notin S_p \cap S_o } b \cdot
  cost_1(scan_d)$, where $b$ is the sharing benefit and
  $cost_1(scan_d)$ is the cost for the first read of the source scan
  operator for source $d$, as introduced in Sec.~\ref{sec:ops}.
\end{definition}

With the maximal benefit, we again define a more strict dominance
relation $>_b$ using the maximal benefit:

\begin{theorem}
  Given a query $Q$, a subexpression $T \subset Q$ and two plans
  $p_1,p_2 \in P(T)$, then $p_1 \mbox{\textnormal{ strictly dominates
    }} p_2$ ($p_1 >_b p_2$) if $card(p_1) \geq card(p_1) \wedge
  cost(p_1) \leq cost(p_2) - m_{p_1}(p_2) \wedge (card(p_1) >
  card(p_2) \vee cost(p_1) < cost(p_2) - m_{p_1}(p_2))$.

  Whenever $p_1$ strictly dominates $p_2$ then $p_1$ dominates $p_2$:
  $p_1 >_b p_2 \Rightarrow p_1 > p_2$.
\end{theorem}
\begin{proof}
  
\end{proof}

With this theorem we can use the maximal benefit to perform pruning
and be sure that no optimal plan is discarded.

\subsection{Pareto-optimality}
\label{sec:pareto}

The goal of the optimizer in the Linked Data query processing scenario
is to find the overall skyline (or Pareto set) of query plans, while
pruning as many plans as possible at each step. We will show that we
can retain only non-dominated plans for each subexpression and still
find the complete overall skyline, i.e., we show that the optimization
problem still has optimal substructure.

The goal is to find the optimal solution for query $Q$, in this case
the optimal solution is the set of Pareto-optimal query plans $P^+(Q)
= P^*(Q)$. To prove that the problem has optimal substructure, we need
to show that we can construct $P^*(Q)$ by splitting $Q$ into
subproblems $T \subset Q$ and then combining optimal solutions
$P^*(T)$ for the subproblems. In particular, this means that a
non-optimal solution for a subproblem must not be part of an optimal
solution for a superproblem.

% The smallest subproblem is a singleton subset of $Q$, i.e., a single
% triple pattern $t$. An optimal solution for this problem is a set of
% Pareto-optimal plans $P^*(\{t\})$. Each such plan $p_t$ is an access
% plan, consisting of a set of source scan operators, whose output is
% fed into separate selection operators and then into a single union
% operator.

% The first step where solutions for subproblems are combined is
% constructing query plans for 2-element subsets of $Q$, i.e., joins
% between base inputs. We show that we can construct $P^*(\{t_1,t_2\})$
% by combining optimal solutions for $t_1$ and $t_2$, i.e.,
% $P^*(\{t_1\})$ and $P^*(\{t_2\})$, for all $t_1,t_2 \in Q$ with $t_1
% \neq t_2$. Let $S$ be the set of all plans constructed by combining
% Pareto-optimal plans for $t_1$ and $t_2$: $S =\{\mathtt{cmb}(p_1,p_2)
% | p_1 \in P^*(t_1), p_2 \in P^*(t_2)\}$. Then we need to show that
% $P^*(\{t_1,t_2\}) = S^*$ or $P^*(\{t_1,t_2\}) \subseteq S$.

% \begin{align*}
%   P^*(\{t_1,t_2\}) \subseteq S &  \Leftrightarrow  \forall p^* \in
%   P^*(\{t_1,t_2\}) : p^* \in S \\
% &   \Leftrightarrow \forall p^* \in P^*(\{t_1,t_2\}) :
%   p^* = \mathtt{cmb}(p^*_1, p^*_2) \;\mathrm{with}\; p^*_1 \in
%   P^*(t_1), p^*_2 \in P^*(t_2)
% \end{align*}

% This means that all optimal, i.e., non-dominated, plans for $t_1,t_2$
% are a combination of non-dominated plans for $t_1$ and $t_2$,
% respectively.

% \todo{benutz kosten ohne bounds hier, oben dann beweisen dass wir beim
%   benutzen der bounds beim prunen irgendwas klappt}

\begin{theorem}
  Given a query $Q$ and two subexpressions $T_1,T_2 \subseteq Q$ with
  $T_1 \cap T_2 = \emptyset$, the set of optimal plans for $T_1 \cap
  T_2$ can be constructed from optimal plans for $T_1,T_2$, i.e.,
  $P^+(T_1 \cup T_2) = P^*(T_1 \cup T_2) \subseteq C =
  \{\mathtt{cmb}(p_1,p_2) | p_1 \in P^*(T_1), p_2 \in P^*(T_2)\}$.
\end{theorem}
\begin{proof}
  Let $p^* \in P^*(T_1 \cup T_2)$ be a plan that is a combination of a
  dominated plan for $T_1$ and a non-dominated plan for $T_2$, i.e.,
  $p^* = \mathtt{cmb}(p^-_1,p^*_2),p^-_1 \in P^-(T_1),p^*_2 \in
  P^*(T_2)$. This means that there is a non-dominated plan $p^*_1 \in
  P^*(T_1)$ that dominates $p^-_1$, but the combination of $p^*_1$
  with $p^*_2$ is dominated by the combination of $p^-_1$ and $p^*_2$:
  \[ \exists p^*_1 \in P^*(T_1) : \mathtt{cmb}(p^-_1,p^*_2) \text{
    dominates } \mathtt{cmb}(p^*_1,p^*_2)\] As $p^*_1$ dominates
  $p^-_1$ and $\mathtt{cmb}(p^-_1,p^*_2)$ dominates
  $\mathtt{cmb}(p^*_1,p^*_2)$, according to
  Def.~\ref{def:dominates_bound}, this implies (without loss of
  generality we assume that both objectives are strictly
  lesser/greater):
  \begin{eqnarray*}
    card(p^-_1) < card(p^*_1) &\wedge& card (\mathtt{cmb}(p^-_1,p^*_2)) > card(\mathtt{cmb}(p^*_1,p^*_2)) \\
    cost^Q_L(p^-_1) < cost^Q_U(p^*_1) &\wedge& cost^Q_U(\mathtt{cmb}(p^-_1,p^*_2)) > cost^Q_L(\mathtt{cmb}(p^*_1,p^*_2))\\
  \end{eqnarray*}
  However, with the montonicity of both, the cost and the cardinality,
  we have a contradiction, because $cost^Q_L(p^-_1) <
  cost^Q_U(p^+_1)$, but $cost^Q_U(\mathtt{cmb}(p^-_1,p^*_2)) >
  cost^Q_L(\mathtt{cmb}(p^*_1,p^*_2))$ (the same holds for the
  cardinality). With regard to our original proposition, this means
  that there is no plan $p^* \in P^*(T_1 \cup T_2)$, such that $p^*$
  is a combination of dominated plan $p^-_1$ and a non-dominated plan
  $p^*_2$. The same holds when $p^*$ is a combination of two dominated
  plans (we omit the proof for brevity). From this follows that all
  $p^* \in P^*(T_1 \cup T_2)$ are combinations of the non-dominated
  plans in $P^*(T_1)$ and $P^*(T_2)$ and therefore $P^*(T_1 \cup T_2)
  \subseteq C$.
\end{proof}

% Proof by contradiction: Suppose there is a non-dominated plan $p^* \in
% P^*(\{t_1,t_2\})$ that is a combination of a dominated plan for $t_1$ and
% a non-dominated plan for $p_2$:
% \[\exists p^* \in P^*(\{t_1,t_2\}) : p^* = \mathtt{cmb}(p^-_1,p^*_2),
% p^-_1 \in P^-(t_1), p^*_2 \in P^*(t_2) \] 

% This would mean that there is a non-dominated plan $p^*_1 \in
% P^*(t_1)$ that naturally dominates $p^-_1$, but the combination of
% $p^*_1$ with $p^*_2$ is dominated by the combination of $p^-_1$ and
% $p^*_2$:
% \[ \exists p^*_1 \in P^*(t_1) : \mathtt{cmb}(p^-_1,p^*_2)
% \text{ dominates } \mathtt{cmb}(p^*_1,p^*_2)\]

% This implies that the score and cardinality of
% $\mathtt{cmb}(p^-_1,p^*_2)$ are greater or equal than the score and
% cardinality of $\mathtt{cmb}(p^*_1,p^*_2)$, whereas the score and
% cardinality of $p^-_1$ is less than the score and cardinality of
% $p^*_1$ (because $p^*_1$ dominates $p^-_1$):

% \begin{eqnarray*}
%   score(p^-_1) < score(p^*_1) &\wedge& score(\mathtt{cmb}(p^-_1,p^*_2)) \geq score(\mathtt{cmb}(p^*_1,p^*_2))\\
%   card(p^-_1) < card(p^*_1) &\wedge& card (\mathtt{cmb}(p^-_1,p^*_2)) \geq card(\mathtt{cmb}(p^*_1,p^*_2)) \\
% \end{eqnarray*}

% However, we postulate that $card$ and $score$ are monotonic with
% regard to the combination of plans:
% \begin{eqnarray*}
%   score(p_1) \leq score(p'_1) &\Rightarrow& score(\mathtt{cmb}(p_1,p_2))
%   \leq score(\mathtt{cmb}(p'_1,p_2)) \\
%   card(p_1) \leq card(p'_1) &\Rightarrow& card(\mathtt{cmb}(p_1,p_2))
%   \leq card(\mathtt{cmb}(p'_1,p_2)) \\
% \end{eqnarray*}

% where $p_1,p'_1$ are any plans for the same expression and $p_2$ is a
% plan for a different, disjunct expression. 

% \todo{add bounds here}

% With the monotonicity of the $score$ and $card$ functions we have a
% contradiction, because $score(p^-_1) < score(p^*_1)$, but
% $score(\mathtt{cmb}(p^-_1,p^*_2)) \geq
% score(\mathtt{cmb}(p^*_1,p^*_2))$ (the same holds for the
% cardinality). With regard to our original proposition, this means that
% there is no plan $p^* \in P^*_{t_1,t_2}$ such that $p^*$ is a
% combination of a dominated plan $p^-_1$ and a non-dominated plan
% $p^*_2$. \todo{show that this also holds when both plans are
%   dominated} From this follows that all $p^* \in P^*_{t_1,t_2}$ are
% combinations of the non-dominated plans in $P^*_{t1}$ and $P^*_{t_2}$.


Although the relaxed comparability constraint allows the optimizer to
prune more aggressively than otherwise possible, we observe that the
goal of generating Pareto-optimal query plans again leads to a larger
search space.

\subsection{Optimizer Algorithm}

With the relaxed comparability constraint and the Pareto-optimality
in place, we can now present the query optimization algorithm for
constructing Pareto-optimal Linked Data query plans.

% \subsubsection{Pareto-optimal Access Plans for Triple Patterns}

% The first step in the dynamic programming algorithm for query
% optimization is creating Pareto-optimal access plans for single query
% triple patterns. As described in Section~\ref{sec:basicshape} these
% plans each consist of a set source scan operators whose output feeds
% into selection operators and then in a single union operator. Given a
% triple pattern $t$, we first use the source index to obtain the set of
% relevant sources $D = I(t)$.

% A naive solution to obtain the Pareto-optimal plans for $t$ would be
% to first create the set of all possible plans and then prune all
% dominated plans. However, we can construct a valid plan for $t$ from
% any subset of $D$, i.e., there is one possible plan for each element
% of the power set of $D$. As the size of the power set is $2^{|D|}$
% this is infeasible even for triple patterns that appear in only
% relatively few sources.

% \todo{add short description of the algorithm we use}

% \subsubsection{Pareto-optimal Dynamic Programming}

\begin{algorithm}
  \label{alg:plan}
  \DontPrintSemicolon

  \SetKwData{BestPlans}{bestPlans}\SetKwData{Null}{null}
  \SetKwFunction{Combine}{combine}

  \caption{\textsc{PlanGen}$(q)$}
  \KwIn{Query $q = \{t_1,\ldots,t_n\}$, Source index $I$}
  \KwOut{Pareto-optimal query plans \BestPlans{q}}

  \ForEach{$t \in q$}{
    $S \leftarrow \{ \cup(\{ \sigma_t(scan(d)) | d \in D \}) | D \in \mathcal{P}(I(t))\}$
%    $S \leftarrow \{\cup(C) | C \in \mathcal{P}(\{\sigma_t(scan(d)) | d \in I(t)\})\}$ \;
    \BestPlans{$\{t\}$}$\leftarrow \{p \in S| \nexists p' \in S : p'
    \mbox{ dominates } p \}$\;
  }

  \For{$i \leftarrow 2$ \KwTo $|q|$}{
    \ForEach{$T \subseteq q$ such that $|T| = i$}{
      \ForEach{$t \subset T$}{
        $S \leftarrow S \cup$ \Combine{\BestPlans{$\{t\}$},
          \BestPlans{$T \setminus \{t\}$}}\;
      }
      \BestPlans{$T$}$\leftarrow \{p \in S | \nexists p' \in S : p'
      \mbox{ dominates } p \}$\;
    }
  }
  \Return \BestPlans{q}
\end{algorithm}

\todo{show generation of access plans}

Algorithm~\ref{alg:plan} shows the method \textsc{PlanGen} that takes
a query $q=\{t_1,\ldots,t_n\}$ as input and returns the Pareto-optimal
plans to evaluate the query. During optimization, \textsf{bestPlans}
stores the best plans for all subsets of $q$. 

In the first step, the plans for single triple patterns are
created. For each triple pattern $t$ in $q$, first the relevant
sources are selected with help of the source index $I$. As we need to
consider all possible combinations of sources, we create the power set
$\mathcal{P}(I(t))$ of all sources. For each member $D$ of the power
set, we create an access plan, consisting of a selection and scan
operator $\sigma_t(scan(d))$ for each source $d \in D$ and a single
union operator $\cup$ that has the selection operators as input. $S$
then contains a set of access plans, one for each combination of
relevant sources. From this set, we then select only the
non-dominated, i.e., Pareto-optimal, access plans and store them in
\textsf{bestPlans}$(\{t\})$.

During the next iterations, joins between previously created plans are
calculated until all query triple patterns are covered. For iteration
$i$, we select all subsets $T \subset q$ with $|T|=i$. For each $t \in
T$ the algorithm creates all possible joins between the best, i.e.,
Pareto-optimal, plans for $t$ and $T\setminus \{t\}$. By selecting
only a single triple pattern from $T$ we create only left-deep plans,
but bushy plans would also be possible.  All these plans are stored in
$S$ and are comparable since they cover the same triple patterns
$T$. Finally, only the non-dominated plans from $S$ are selected and
stored in \textsf{bestPlans}$(T)$. After the last iteration,
\textsf{bestPlans}$(q)$ contains the Pareto-optimal left-deep plans
for $q$.

\todo{describe combine, duplicate source scan operators are combined
  here; description of cost calculation for such a DAG is in earlier
  section}


% \subsection{Adaptivity}
% The goal of adaptive query processing is to perform query optimization
% in the case were no complete knowledge is available and adapt the
% query processing at run-time by using newly available
% knowledge. During processing of Linked Data queries, new information
% about sources becomes available: 1) new sources may be discovered, 2)
% state accumulated inside join operators can be used to perform better
% estimates of join cardinalities for a particular source and 3) data
% properties may be observer to deviate from previous estimates.

% Here, we adopt techniques from previous research on adaptive query
% processing.

% \todo{what do we monitor? when is re-optimization done?}

% Linked Data query processing requires ranking to be performed not only
% at compile-time, but also continously at run-time in order to take
% advantage of knowledge gained during query processing. Using
% adaptive query processing techniques not only ranking can be
% performed, but full query optimization at run-time. 

% Weight distribution between cost and relevancy might change over time:
% at first, relevancy is more important, while later on, after results
% have been produced, cost becomes more important?


% \subsection{Implementation}
% \label{sec:impl}

% \todo{describe concrete implementation, in particular methods used for
% cost/result estimation and query strategies different from the optimal
% ones}

% \todo{dependencies between sources are not reflected by available
%   statistics, but indirectly captured using run-time reestimation of
%   result sizes using sampling etc.}



%%% Local Variables: 
%%% mode: latex
%%% TeX-master: "paper"
%%% End: 
