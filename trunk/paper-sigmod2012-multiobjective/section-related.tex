\section{Related Work}
\label{sec:related}

\textbf{Linked Data Query Processing.} The concept of executing SPARQL
queries directly over Linked Data instead of a locally indexed copy
was first introduced in \cite{hartig_executing_2009}, where link
traversal is used to discover sources at runtime. In
\cite{harth_data_2010} a local source index based on QTrees is used to
speed up the discovery of relevant sources. Our previous work
\cite{ladwig_linked_2010,sihjoin_2011} proposes methods for
ranking sources at runtime according to their relevancy and a mixed
execution strategy that combines both, link traversal and source
indexes, to report results early. In this work we adopt the approach
from \cite{harth_data_2010} and employ a source index without any
runtime source discovery along with the push-based Symmetric Hash Join
operator from \cite{ladwig_linked_2010,sihjoin_2011}.


% \textbf{Dynamic Programming.}

% Faster enumeration of plans: \cite{moerkotte_analysis_2006,moerkotte_dynamic_2008}

% Iterative Dynamic Programming (approx): \cite{kossmann_iterative_2000}

\textbf{Multiobjective Query Optimization.} Multi-objective query
optimization was previously proposed in
\cite{papadimitriou_multiobjective_2001}, where it is discussed in the
context of Mariposa, a wide-area database system. The Mariposa
optimizer splits the query tree into subqueries and then obtains
\emph{bids} from participating database sites that specify a delay and
cost for delivering the result of a subquery. The goal of the optimzer
is to obtain the Pareto optimal set of plans with respect to cost and
delay. While the authors do employ dynamic programming to show that
the Pareto set can be approximated in polynomial time, it is not based
on the classic dynamic programming algorithm for query optimization
proposed in \cite{selinger_access_1979}. The problem is slightly
different as there is only a single query operation tree and for each
operation node the optimizer is provided a list of alternatives for
implementing the operation. In contrast, the classic dynamic
programming \cite{selinger_access_1979} does not consider only a
single query tree (and therefore a single order of operations), but
builds and optimizes physical query plans from the bottom up and
considers all valid query trees. In this work we extend the classic
algorithm to support multi-objective query optimization.


The approach presented in \cite{nie_joint_2001} optimizes query plans
not only for cost, but also for coverage (i.e. output cardinality) and
thereby integrates source selection into the query optimization
process. However, the optimization is performed by combining
(weighted) cost and coverage into a utility function that provides a
single measure to assess query plans. Using a single measure means
that traditional query optimization algorithms, such as dynamic
programming \cite{selinger_access_1979}, are directly
applicable. However, no true multi-objective optimization is
performed.


\textbf{Data Integration.}

Bucket algorithm: \cite{levy_querying_1996}

MiniCon algorithm: \cite{pottinger_minicon:_2001}

Optimization of cost and coverage (utility function):
\cite{nie_joint_2001}

Source selection: \cite{pomares_source_2010}

Query Planning in the Presence of Overlapping Sources:
\cite{bleiholder_query_2006}

survey materialized views: \cite{halevy_answering_2001}



%%% Local Variables: 
%%% mode: latex
%%% TeX-master: "paper"
%%% End: 
